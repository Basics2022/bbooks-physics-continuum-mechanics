%% Generated by Sphinx.
\def\sphinxdocclass{jupyterBook}
\documentclass[letterpaper,10pt,english]{jupyterBook}
\ifdefined\pdfpxdimen
   \let\sphinxpxdimen\pdfpxdimen\else\newdimen\sphinxpxdimen
\fi \sphinxpxdimen=.75bp\relax
\ifdefined\pdfimageresolution
    \pdfimageresolution= \numexpr \dimexpr1in\relax/\sphinxpxdimen\relax
\fi
%% let collapsible pdf bookmarks panel have high depth per default
\PassOptionsToPackage{bookmarksdepth=5}{hyperref}
%% turn off hyperref patch of \index as sphinx.xdy xindy module takes care of
%% suitable \hyperpage mark-up, working around hyperref-xindy incompatibility
\PassOptionsToPackage{hyperindex=false}{hyperref}
%% memoir class requires extra handling
\makeatletter\@ifclassloaded{memoir}
{\ifdefined\memhyperindexfalse\memhyperindexfalse\fi}{}\makeatother

\PassOptionsToPackage{warn}{textcomp}

\catcode`^^^^00a0\active\protected\def^^^^00a0{\leavevmode\nobreak\ }
\usepackage{cmap}
\usepackage{fontspec}
\defaultfontfeatures[\rmfamily,\sffamily,\ttfamily]{}
\usepackage{amsmath,amssymb,amstext}
\usepackage{polyglossia}
\setmainlanguage{english}



\setmainfont{FreeSerif}[
  Extension      = .otf,
  UprightFont    = *,
  ItalicFont     = *Italic,
  BoldFont       = *Bold,
  BoldItalicFont = *BoldItalic
]
\setsansfont{FreeSans}[
  Extension      = .otf,
  UprightFont    = *,
  ItalicFont     = *Oblique,
  BoldFont       = *Bold,
  BoldItalicFont = *BoldOblique,
]
\setmonofont{FreeMono}[
  Extension      = .otf,
  UprightFont    = *,
  ItalicFont     = *Oblique,
  BoldFont       = *Bold,
  BoldItalicFont = *BoldOblique,
]



\usepackage[Bjarne]{fncychap}
\usepackage[,numfigreset=1,mathnumfig]{sphinx}

\fvset{fontsize=\small}
\usepackage{geometry}


% Include hyperref last.
\usepackage{hyperref}
% Fix anchor placement for figures with captions.
\usepackage{hypcap}% it must be loaded after hyperref.
% Set up styles of URL: it should be placed after hyperref.
\urlstyle{same}

\addto\captionsenglish{\renewcommand{\contentsname}{Continuum Mechanics}}

\usepackage{sphinxmessages}



        % Start of preamble defined in sphinx-jupyterbook-latex %
         \usepackage[Latin,Greek]{ucharclasses}
        \usepackage{unicode-math}
        % fixing title of the toc
        \addto\captionsenglish{\renewcommand{\contentsname}{Contents}}
        \hypersetup{
            pdfencoding=auto,
            psdextra
        }
        % End of preamble defined in sphinx-jupyterbook-latex %
        

\title{continuum mechanics}
\date{Jan 15, 2025}
\release{}
\author{basics}
\newcommand{\sphinxlogo}{\vbox{}}
\renewcommand{\releasename}{}
\makeindex
\begin{document}

\pagestyle{empty}
\sphinxmaketitle
\pagestyle{plain}
\sphinxtableofcontents
\pagestyle{normal}
\phantomsection\label{\detokenize{intro::doc}}


\sphinxAtStartPar
\sphinxstylestrong{Cinematica.}
Descrizione lagrangiana, euleriana e arbitraria.

\sphinxAtStartPar
\sphinxstylestrong{Equazioni di bilancio per mezzi continui.}
\begin{itemize}
\item {} 
\sphinxAtStartPar
Princìpi (massa, quantità di moto, energia totale):
\begin{itemize}
\item {} 
\sphinxAtStartPar
Forma integrale:
\begin{itemize}
\item {} 
\sphinxAtStartPar
volume materiale, sistema chiuso

\item {} 
\sphinxAtStartPar
volumi arbitrari (tramite derivazione su domini mobili, teoremi del trasporto \sphinxstylestrong{todo} link)

\end{itemize}

\item {} 
\sphinxAtStartPar
Forma differenziale:
\begin{itemize}
\item {} 
\sphinxAtStartPar
coordinate fisiche, materiali e arbitrarie

\end{itemize}

\end{itemize}

\end{itemize}

\sphinxAtStartPar
\sphinxstylestrong{Equazioni di stato ed equazioni costitutive.}
\begin{itemize}
\item {} 
\sphinxAtStartPar
legami tra variabili termodinamiche

\item {} 
\sphinxAtStartPar
espressione del tensore degli sforzi e del flusso di conduzione

\end{itemize}

\sphinxAtStartPar
\sphinxstylestrong{Bilanci derivati.}
\begin{itemize}
\item {} 
\sphinxAtStartPar
energia cinetica

\item {} 
\sphinxAtStartPar
energia interna

\item {} 
\sphinxAtStartPar
entropia

\end{itemize}

\sphinxstepscope


\part{Continuum Mechanics}

\sphinxstepscope


\chapter{Cinematica}
\label{\detokenize{ch/continuum/kinematics:cinematica}}\label{\detokenize{ch/continuum/kinematics:continuum-kinematics}}\label{\detokenize{ch/continuum/kinematics::doc}}\begin{itemize}
\item {} 
\sphinxAtStartPar
coordinate fisiche \(\mathbf{r}\)

\item {} 
\sphinxAtStartPar
coordinate materiali \(\mathbf{r}_0\)

\item {} 
\sphinxAtStartPar
coordinate arbitrarie \(\mathbf{r}_b\)

\end{itemize}

\sphinxAtStartPar
Il moto dei punti materiali nello spazio fisico è descritto dalla una funzione
\begin{equation*}
\begin{split}\mathbf{r}(\mathbf{r}_0, t) \ ,\end{split}
\end{equation*}
\sphinxAtStartPar
che associa a ogni punto materiale “etichettato” con la coordinata \(\mathbf{r}_0\) la sua posizione nello spazio al tempo \(t\)

\sphinxAtStartPar
Il campo di velocità dei punti materiali è definito come la derivata nel tempo della posizione, tenendo fissa la coordinata \(\mathbf{r}_0\) che identifica i punti,
\begin{equation*}
\begin{split}\mathbf{u}(\mathbf{r}_0,t) = \dfrac{\partial \mathbf{r}}{\partial t}\bigg|_{\mathbf{r}_0} = \dfrac{D \mathbf{r}}{D t} \ ,\end{split}
\end{equation*}
\sphinxAtStartPar
avendo introdotto l’operatore di derivata materiale \(\frac{D}{Dt}\).

\sphinxAtStartPar
Alla stessa maniera la posizione nello spazio e la velocità dei punti in moto arbitrario, etichettati con le coordinate \(\mathbf{r}_b\), sono
\begin{equation*}
\begin{split}\mathbf{r}(\mathbf{r}_b, t) \ ,\end{split}
\end{equation*}\begin{equation*}
\begin{split}\mathbf{u}(\mathbf{r}_b,t) = \dfrac{\partial \mathbf{r}}{\partial t}\bigg|_{\mathbf{r}_b} \ .\end{split}
\end{equation*}
\sphinxAtStartPar
Le leggi di trasformazione tra le coordinate forniscono anche la legge di trasformazione delle derivate parziali,


\begin{equation*}
\begin{split}\begin{aligned}
\dfrac{\partial}{\partial t}\bigg|_{\mathbf{r}_0} f(\mathbf{r}(\mathbf{r}_0, t), t) 
    = \dfrac{\partial f}{\partial t}\bigg|_{\mathbf{r}} +
    \dfrac{\partial \mathbf{r}}{\partial t}\bigg|_{\mathbf{r}_0} \cdot \dfrac{\partial f}{\partial \mathbf{r}}\bigg|_{t} 
    = \dfrac{\partial f}{\partial t}\bigg|_{\mathbf{r}} +
      \mathbf{u} \cdot \nabla f 
\end{aligned}\end{split}
\end{equation*}\begin{equation*}
\begin{split}\begin{aligned}
\dfrac{\partial}{\partial t}\bigg|_{\mathbf{r}_b} f(\mathbf{r}(\mathbf{r}_b, t), t) 
    = \dfrac{\partial f}{\partial t}\bigg|_{\mathbf{r}} +
    \dfrac{\partial \mathbf{r}}{\partial t}\bigg|_{\mathbf{r}_b} \cdot \dfrac{\partial f}{\partial \mathbf{r}}\bigg|_{t} 
    = \dfrac{\partial f}{\partial t}\bigg|_{\mathbf{r}} +
      \mathbf{u}_b \cdot \nabla f 
\end{aligned}\end{split}
\end{equation*}
\sphinxAtStartPar
e quindi
\begin{equation*}
\begin{split}
\dfrac{\partial}{\partial t}\bigg|_{\mathbf{r}_0} f =  
\dfrac{\partial}{\partial t}\bigg|_{\mathbf{r}  } f + \mathbf{u} \cdot \nabla f =  
\dfrac{\partial}{\partial t}\bigg|_{\mathbf{r}_b} f + ( \mathbf{u} - \mathbf{u}_b ) \cdot \nabla f \ . 
\end{split}
\end{equation*}
\sphinxAtStartPar
()=


\section{Kinematics of two points}
\label{\detokenize{ch/continuum/kinematics:kinematics-of-two-points}}\begin{equation*}
\begin{split}\vec{r}_2(t) - \vec{r}_1(t) = \dots \end{split}
\end{equation*}
\sphinxAtStartPar
strain velocity tensor
\begin{equation}\label{equation:ch/continuum/kinematics:eq:strain-vel-tensor}
\begin{split}\mathbb{D} = \frac{1}{2} \left[ \nabla \vec{u} + \nabla^T \vec{u} \right]\end{split}
\end{equation}
\sphinxstepscope


\chapter{Princìpi in forma integrale per volumi materiali}
\label{\detokenize{ch/continuum/principles-integral:principi-in-forma-integrale-per-volumi-materiali}}\label{\detokenize{ch/continuum/principles-integral:continuum-principles-integral}}\label{\detokenize{ch/continuum/principles-integral::doc}}
\sphinxAtStartPar
In meccanica classica, per sistemi chiusi

\sphinxAtStartPar
\sphinxstylestrong{Principio di conservazione della massa: bilancio di massa.} \(\dot{M}_{V_t} = 0\)
\begin{equation*}
\begin{split}\frac{d}{dt} \int_{V_t} \rho = 0 \ .\end{split}
\end{equation*}
\sphinxAtStartPar
\sphinxstylestrong{Secondo principio della dinamica: bilancio di quantità di moto.} \(\dot{\mathbf{Q}} = \mathbf{R}^{ext}\)
\begin{equation*}
\begin{split}\dfrac{d}{dt} \int_{V_t} \rho \mathbf{u} = \int_{V_t} \rho \mathbf{g} + \oint_{\partial V_t} \mathbf{t}_{\mathbf{n}} \ .\end{split}
\end{equation*}
\sphinxAtStartPar
\sphinxstylestrong{Primo principio della termodinamica: bilancio di energia totale.} \(\dot{E}^{tot} = P^{ext} + \dot{Q}^{ext}\)
\begin{equation*}
\begin{split}\dfrac{d}{dt} \int_{V_t} \rho e^t = \int_{V_t} \rho \mathbf{g} \cdot \mathbf{u} + \oint_{\partial V_t} \mathbf{t}_{\mathbf{n}} \cdot \mathbf{u} - \oint_{\partial V_t} \mathbf{q} \cdot \mathbf{\hat{n}} + \int_{V_t} \rho r \ .\end{split}
\end{equation*}
\sphinxstepscope


\chapter{Princìpi in forma integrale per domini arbitrari}
\label{\detokenize{ch/continuum/principles-integral-arbitrary:principi-in-forma-integrale-per-domini-arbitrari}}\label{\detokenize{ch/continuum/principles-integral-arbitrary:continuum-principles-integral-arbitrary}}\label{\detokenize{ch/continuum/principles-integral-arbitrary::doc}}
\sphinxAtStartPar
Usando i teoremi del trasporto per la derivata nel tempo di grandezze fisiche su domini mobili \sphinxstylestrong{todo} \sphinxstyleemphasis{aggiungere riferimento} si può ricavare la forma integrale dei princìpi per sistemi aperti.


\section{Volume di controllo, \protect\(V\protect\)}
\label{\detokenize{ch/continuum/principles-integral-arbitrary:volume-di-controllo-v}}
\sphinxAtStartPar
\sphinxstylestrong{Bilancio di massa}
\begin{equation*}
\begin{split}\frac{d}{dt} \int_{V} \rho + \oint_{\partial V} \rho \mathbf{u} \cdot \mathbf{\hat{n}} = 0 \ .\end{split}
\end{equation*}
\sphinxAtStartPar
\sphinxstylestrong{Bilancio della quantità di moto}
\begin{equation*}
\begin{split}\dfrac{d}{dt} \int_{V} \rho \mathbf{u} + \oint_{\partial V} \rho \mathbf{u} \mathbf{u} \cdot \mathbf{\hat{n}} = \int_{V} \rho \mathbf{g} + \oint_{\partial V} \mathbf{t}_{\mathbf{n}} \ .\end{split}
\end{equation*}
\sphinxAtStartPar
\sphinxstylestrong{Bilancio dell’energia totale.}
\begin{equation*}
\begin{split}\dfrac{d}{dt} \int_{V} \rho e^t + \oint_{\partial V} \rho e^t \mathbf{u} \cdot \mathbf{\hat{n}} = \int_{V} \rho \mathbf{g} \cdot \mathbf{u} + \oint_{\partial V} \mathbf{t}_{\mathbf{n}} \cdot \mathbf{u} - \oint_{\partial V} \mathbf{q} \cdot \mathbf{\hat{n}} + \int_{V} \rho r \ .\end{split}
\end{equation*}

\section{Volume arbitrario, \protect\(v_t\protect\)}
\label{\detokenize{ch/continuum/principles-integral-arbitrary:volume-arbitrario-v-t}}
\sphinxAtStartPar
\sphinxstylestrong{Bilancio di massa}
\begin{equation*}
\begin{split}\frac{d}{dt} \int_{v_t} \rho + \oint_{\partial v_t} \rho ( \mathbf{u} - \mathbf{u}_b ) \cdot \mathbf{\hat{n}} = 0 \ .\end{split}
\end{equation*}
\sphinxAtStartPar
\sphinxstylestrong{Bilancio della quantità di moto}
\begin{equation*}
\begin{split}\dfrac{d}{dt} \int_{v_t} \rho \mathbf{u} + \oint_{\partial v_t} \rho \mathbf{u} ( \mathbf{u} - \mathbf{u}_b ) \cdot \mathbf{\hat{n}} = \int_{v_t} \rho \mathbf{g} + \oint_{\partial v_t} \mathbf{t}_{\mathbf{n}} \ .\end{split}
\end{equation*}
\sphinxAtStartPar
\sphinxstylestrong{Primo principio della termodinamica: bilancio di energia totale.} \(\dot{E}^{tot} = P^{ext} + \dot{Q}^{ext}\)
\begin{equation*}
\begin{split}\dfrac{d}{dt} \int_{v_t} \rho e^t + \oint_{\partial v_t} \rho e^t ( \mathbf{u} - \mathbf{u}_b ) \cdot \mathbf{\hat{n}} = \int_{v_t} \rho \mathbf{g} \cdot \mathbf{u} + \oint_{\partial v_t} \mathbf{t}_{\mathbf{n}} \cdot \mathbf{u} - \oint_{\partial v_t} \mathbf{q} \cdot \mathbf{\hat{n}} + \int_{v_t} \rho r \ .\end{split}
\end{equation*}
\sphinxstepscope


\chapter{Princìpi in forma differenziale}
\label{\detokenize{ch/continuum/principles-differential:principi-in-forma-differenziale}}\label{\detokenize{ch/continuum/principles-differential:continuum-principles-differential}}\label{\detokenize{ch/continuum/principles-differential::doc}}

\section{Bilanci in coordinate fisiche}
\label{\detokenize{ch/continuum/principles-differential:bilanci-in-coordinate-fisiche}}
\sphinxAtStartPar
Partendo dai bilanci in forma integrale per volumi di controllo, se sono soddifatte le condizioni di “sufficiente regolarità” dei campi (intese come “ogni cosa che scriviamo ha senso”), si possono ricavare i bilanci i forma differenziale, sfruttando il teorema della divregenza per trasformare gli integrali di superficie in integrali di volume, e l’arbitrarietà del volume di controllo (poiché la validità dei princìpi fisici è indipendente dal particolare sistema considerato).

\sphinxAtStartPar
Viene usata la relazione di Cauchy per esprimere il vettore sforzo sul contorno del dominio \(\partial V\) come in funzione della normale alla superficie e del tensore degli sforzi \(\mathbb{T}\) \sphinxstylestrong{todo} riferimento alla relazione di Cauchy,
\begin{equation*}
\begin{split}\mathbf{t_n} = \mathbf{\hat{n}} \cdot \mathbb{T} \ .\end{split}
\end{equation*}
\sphinxAtStartPar
\sphinxstylestrong{Bilancio di massa.}
\begin{equation*}
\begin{split}\begin{aligned}
 0 & = \frac{d}{dt} \int_{V} \rho + \oint_{\partial V} \rho \mathbf{u} \cdot \mathbf{\hat{n}} = \\
   & = \int_{V} \dfrac{\partial \rho }{\partial t} + \int_{V} \nabla \cdot \left( \rho \mathbf{u} \right) = \\
   & = \int_{V} \left\{ \dfrac{\partial \rho }{\partial t} + \nabla \cdot \left( \rho \mathbf{u} \right) \right\} 
\end{aligned}\end{split}
\end{equation*}\begin{equation*}
\begin{split}\dfrac{\partial \rho }{\partial t} + \nabla \cdot \left( \rho \mathbf{u} \right) = 0\end{split}
\end{equation*}
\sphinxAtStartPar
\sphinxstylestrong{Bilancio della quantità di moto.}
\begin{equation*}
\begin{split}\begin{aligned}
 \mathbf{0} & = \dfrac{d}{dt} \int_{V} \rho \mathbf{u} + \oint_{\partial V} \rho \mathbf{u} \mathbf{u} \cdot \mathbf{\hat{n}} - \int_{V} \rho \mathbf{g} - \oint_{\partial V} \mathbf{\hat{n}} \cdot \mathbb{T} = \\
            & = \int_{V} \dfrac{\partial }{\partial t} \left( \rho \mathbf{u} \right) + \int_{V} \nabla \cdot \left( \rho \mathbf{u} \otimes \mathbf{u} \right) - \int_{V} \rho \mathbf{g} - \int_{V} \nabla \cdot \mathbb{T} = \\
            & = \int_{V} \left\{ \dfrac{\partial }{\partial t} \left( \rho \mathbf{u} \right) +  \nabla \cdot \left( \rho \mathbf{u} \otimes \mathbf{u} \right) - \rho \mathbf{g} - \nabla \cdot \mathbb{T} \right\} 
\end{aligned}\end{split}
\end{equation*}\begin{equation*}
\begin{split} \dfrac{\partial }{\partial t} \left( \rho \mathbf{u} \right) +  \nabla \cdot \left( \rho \mathbf{u} \otimes \mathbf{u} \right) = \rho \mathbf{g} - \nabla \cdot \mathbb{T} \end{split}
\end{equation*}
\sphinxAtStartPar
\sphinxstylestrong{Bilancio dell’energia totale.}


\begin{equation*}
\begin{split}\begin{aligned}
0 & = \dfrac{d}{dt} \int_{V} \rho e^t + \oint_{\partial V} \rho e^t \mathbf{u} \cdot \mathbf{\hat{n}} - \int_{V} \rho \mathbf{g} \cdot \mathbf{u} - \oint_{\partial V} \mathbf{\hat{n}} \cdot \mathbb{T} \cdot \mathbf{u} + \oint_{\partial V} \mathbf{q} \cdot \mathbf{\hat{n}} - \int_{V} \rho r = \\
  & = \int_{V} \dfrac{\partial}{\partial t}  \left( \rho e^t \right) + \int_{\partial V} \nabla \cdot \left( \rho e^t \mathbf{u} \right) - \int_{V} \rho \mathbf{g} \cdot \mathbf{u} - \int_{V} \nabla \cdot \left( \mathbb{T} \cdot \mathbf{u} \right) + \int_{V} \nabla \cdot \mathbf{q} - \int_{V} \rho r = \\
  & = \int_{V} \left\{ \dfrac{\partial}{\partial t}  \left( \rho e^t \right) + \nabla \cdot \left( \rho e^t \mathbf{u} \right) - \rho \mathbf{g} \cdot \mathbf{u} - \nabla \cdot \left( \mathbb{T} \cdot \mathbf{u} \right) + \nabla \cdot \mathbf{q} - \rho r \right\} 
\end{aligned}\end{split}
\end{equation*}\begin{equation*}
\begin{split}
  \dfrac{\partial}{\partial t}  \left( \rho e^t \right) + \nabla \cdot \left( \rho e^t \mathbf{u} \right) = \rho \mathbf{g} \cdot \mathbf{u} + \nabla \cdot \left( \mathbb{T} \cdot \mathbf{u} \right) - \nabla \cdot \mathbf{q} + \rho r
\end{split}
\end{equation*}
\sphinxstepscope


\chapter{Equazioni di stato ed equazioni costitutive}
\label{\detokenize{ch/continuum/constitutive-equations:equazioni-di-stato-ed-equazioni-costitutive}}\label{\detokenize{ch/continuum/constitutive-equations:continuum-constitutive-equations}}\label{\detokenize{ch/continuum/constitutive-equations::doc}}
\sphinxstepscope


\chapter{Equazioni di bilancio di altre grandezze fisiche}
\label{\detokenize{ch/continuum/derived-balances:equazioni-di-bilancio-di-altre-grandezze-fisiche}}\label{\detokenize{ch/continuum/derived-balances:continuum-derived-balances}}\label{\detokenize{ch/continuum/derived-balances::doc}}


\sphinxAtStartPar
Partendo dai bilanci di massa, quantità di moto e di energia totale, si possono ricare le le equazioni di bilancio di altre grandezze fisiche come l’\sphinxstyleemphasis{energia cinetica}, l’\sphinxstyleemphasis{energia interna}, l’\sphinxstyleemphasis{entropia}.


\section{Bilanci in forma differenziale, convettiva}
\label{\detokenize{ch/continuum/derived-balances:bilanci-in-forma-differenziale-convettiva}}
\sphinxAtStartPar
\sphinxstylestrong{Energia cinetica.} L’energia cinetica (macroscopica) per unità di massa è \(k = \frac{1}{2}|\mathbf{u}|^2 = \frac{1}{2} \mathbf{u} \cdot \mathbf{u}\). L’equazione di bilancio dell’energia cinetica viene derivata moltiplicando scalarmente l’equazione della quantità di moto
\begin{equation*}
\begin{split}\rho \dfrac{D \mathbf{u}}{Dt} = \rho \mathbf{g} + \nabla \cdot \mathbb{T} \ ,\end{split}
\end{equation*}
\sphinxAtStartPar
per il campo di velocità \(\mathbf{u}\),
\begin{equation*}
\begin{split}\rho \dfrac{D k}{Dt} = \rho \mathbf{g} \cdot \mathbf{u} + \nabla \cdot \mathbb{T} \cdot \mathbf{u} \ ,\end{split}
\end{equation*}
\sphinxAtStartPar
avendo usato \(\mathbf{u} \cdot d \mathbf{u} = d \left( \dfrac{\mathbf{u} \cdot \mathbf{u}}{2} \right) = dk\).

\sphinxAtStartPar
\sphinxstylestrong{Energia interna.} L’energia interna per unità di massa è la differenza tra l’energia totale e l’energia cinetica, \(e = e^{tot} - k\). L’equazione di bilancio dell’energia interna viene ottenuta come differenza dell’equazione dell’energia totale
\begin{equation*}
\begin{split}\rho \dfrac{D e^{tot}}{Dt} = \rho \mathbf{g} \cdot \mathbf{u} + \nabla \cdot (\mathbb{T} \cdot \mathbf{u}) - \nabla \cdot \mathbf{q} + \rho r \ ,\end{split}
\end{equation*}
\sphinxAtStartPar
e quella dell’energia cinetica, per ottenere
\begin{equation*}
\begin{split}\rho \dfrac{D e}{D t} = \mathbb{T} : \nabla \mathbf{u} - \nabla \cdot \mathbf{q} + \rho r \ . \end{split}
\end{equation*}
\sphinxAtStartPar
\sphinxstylestrong{Entropia.}
\begin{itemize}
\item {} 
\sphinxAtStartPar
\sphinxstylestrong{Entropia nei fluidi.} Se l’entropia può essere scritta come funzione dell’energia interna e della densità, e il primo principio della termodinamica viene scritto come
\begin{equation*}
\begin{split}de = \frac{P}{\rho^2} \, d \rho + T \, ds \ ,\end{split}
\end{equation*}
\sphinxAtStartPar
e il tensore degli sforzi può essere rappresentato come somma degli sforzi di pressione e degli sforzi viscosi \sphinxstylestrong{todo} \sphinxstyleemphasis{riferimento alle leggi costitutive},
\begin{equation*}
\begin{split}\mathbb{T} = - P \mathbb{I} + \mathbb{S} = - P \mathbb{I} + 2 \mu \mathbb{D} + \lambda (\nabla \cdot \mathbf{u}) \mathbb{I} \ ,\end{split}
\end{equation*}
\sphinxAtStartPar
si può ricavare l’equazione di governo dell’entropia usando il differenziale \(ds = \dfrac{1}{T} \, de - \dfrac{P}{T \rho^2} \, d \rho\)
\begin{equation*}
\begin{split}\begin{aligned}
    \rho \dfrac{D s}{D t} 
    & = \dfrac{\rho}{T} \left( \dfrac{D e }{D t} - \dfrac{P}{\rho^2} \dfrac{D \rho}{D t} \right) = \\
    & = \dfrac{1}{T} \left( \rho \dfrac{D e }{D t} - \dfrac{P}{\rho} \dfrac{D \rho}{D t} \right) = \\
    & = \dfrac{1}{T} \left( \mathbb{T} : \nabla \mathbf{u} - \nabla \cdot \mathbf{q} + \rho r - \dfrac{P}{\rho} \left( \rho \nabla \cdot \mathbf{u} \right) \right) = \\
    & = \dfrac{1}{T} \left( -P \nabla \cdot \mathbf{u} + \mathbb{S} : \nabla \mathbf{u} - \nabla \cdot \mathbf{q} + \rho r + P \ \nabla \cdot \mathbf{u}  \right) = \\
    & = \dfrac{1}{T} \left( \mathbb{S} : \nabla \mathbf{u} - \nabla \cdot \mathbf{q} + \rho r \right) = \\
    & = \dfrac{1}{T} \left( 2 \mu |\mathbb{D}|^2 + \lambda |(\nabla \cdot \mathbf{u})|^2 - \nabla \cdot \mathbf{q} + \rho r \right) = \\
    & = \dfrac{2 \mu |\mathbb{D}|^2 + \lambda |(\nabla \cdot \mathbf{u})|^2 }{T} - \frac{\mathbf{q} \cdot \nabla T}{T^2} - \nabla \cdot \left( \dfrac{\mathbf{q}}{T} \right) + \dfrac{\rho r}{T} \ , \\
  \end{aligned}\end{split}
\end{equation*}
\sphinxAtStartPar
avendo usato la degola del prodotto \(\nabla \cdot \left( \dfrac{\mathbf{q}}{T} \right) = \dfrac{\nabla \cdot \mathbf{q}}{T} - \dfrac{\mathbf{q} \cdot \nabla T}{T^2}\).

\sphinxAtStartPar
Gli ultimi due termini sono legati alla \sphinxstylestrong{sorgenti di entropia} nel sistema, dovute alla sorgente di calore nel sistema e al flusso di calore tramite la frontiera del sistema.

\sphinxAtStartPar
I primi due termini possono essere ricondotti alla \sphinxstylestrong{dissipazione viscosa} e dovuta alla \sphinxstylestrong{conduzione termica} all’interno del volume: entrambi devono essere non\sphinxhyphen{}negativi per il secondo principio della termodinamica \sphinxstylestrong{todo}. Il primo termine è positivo se i coefficienti di viscosità del modello di fluido newtoniano sono non\sphinxhyphen{}negativi
\begin{equation*}
\begin{split}\mu, \lambda \ge 0\end{split}
\end{equation*}
\sphinxAtStartPar
. Il secondo termine impone che il flusso di calore avvenga in direzione opposta al gradiente di temperatura locale, e quindi la proiezione su di esso sia negativa (traducendo il concetto che il calore trasferisce energia da un corpo caldo a uno freddo),
\begin{equation*}
\begin{split}- \mathbf{q} \cdot \nabla T \ge 0 \ ,\end{split}
\end{equation*}
\sphinxAtStartPar
come è facile da verificare per il modello di Fourier per la conduzione in mezzi isotropi, \(\mathbf{q} = - k \nabla T\), \(- \mathbf{q} \cdot \nabla T = k |\nabla T|^2 \ge 0\) se
\begin{equation*}
\begin{split}k \ge 0 \ .\end{split}
\end{equation*}
\sphinxAtStartPar
Nel caso di modello lineare per la conduzione in mezzi non isotrpi, il flusso di conduzione può essere descritto usando un tensore del secondo ordine \(\mathbb{K}\), \(\mathbf{q} = - \mathbb{K} \cdot \nabla T\) (\sphinxstylestrong{todo} simmetria?) e la condizione diventa
\begin{equation*}
\begin{split}0 \le - \nabla T  \cdot \mathbf{q} = \nabla T \cdot \mathbb{K} \cdot \nabla T \ ,\end{split}
\end{equation*}
\sphinxAtStartPar
che impone che il tensore di conduzione sia (semi\sphinxhyphen{})definito positivo, a causa dell’arbitrarietà del vettore \(\nabla T\).

\sphinxAtStartPar
Se questi due termini sono non\sphinxhyphen{}negativi, il bilancio di entropia può essere riscritto come la disuguaglianza
\begin{equation*}
\begin{split}\begin{aligned}
    \rho \dfrac{D s}{D t}
    & = \underbrace{\dfrac{2 \mu |\mathbb{D}|^2 + \lambda |(\nabla \cdot \mathbf{u})|^2 }{T} - \frac{\mathbf{q} \cdot \nabla T}{T^2}}_{\ge 0} - \nabla \cdot \left( \dfrac{\mathbf{q}}{T} \right) + \dfrac{\rho r}{T} = \\
    & \ge  - \nabla \cdot \left( \dfrac{\mathbf{q}}{T} \right) + \dfrac{\rho r}{T} \ ,
  \end{aligned}\end{split}
\end{equation*}
\sphinxAtStartPar
o nella forma integrale per un volume materiale
\begin{equation*}
\begin{split}\dfrac{d}{dt} \int_{V_t} \rho s \ge - \oint_{\partial V_t} \mathbf{\hat{n}} \cdot \dfrac{\mathbf{q}}{T} + \int_{V_t} \rho \dfrac{r}{T} \ ,\end{split}
\end{equation*}
\sphinxAtStartPar
che richiama alla mente la disuguaglianza di Clausius \sphinxstylestrong{todo} \sphinxstyleemphasis{aggiungere riferimento}
\begin{equation*}
\begin{split}d S \ge \dfrac{\delta Q^{e}}{T} \ .\end{split}
\end{equation*}
\sphinxAtStartPar
La differenza di segno deriva dalla definizione di \(d Q^e\) come flusso di calore dall’ambiente verso il sistema e del vettore flusso di calore \(\mathbf{q}\) come flusso di calore “uscente dal sistema” \sphinxstylestrong{todo}

\end{itemize}

\sphinxstepscope


\part{Solid Mechanics}

\sphinxstepscope


\chapter{Introduction to Solid Mechanics}
\label{\detokenize{ch/solids/intro:introduction-to-solid-mechanics}}\label{\detokenize{ch/solids/intro:solid-mechanics-intro}}\label{\detokenize{ch/solids/intro::doc}}
\sphinxstepscope


\part{Fluid Mechanics}

\sphinxstepscope


\chapter{Introduction to Fluid Mechanics}
\label{\detokenize{ch/fluids/intro:introduction-to-fluid-mechanics}}\label{\detokenize{ch/fluids/intro:fluid-mechanics-intro}}\label{\detokenize{ch/fluids/intro::doc}}\begin{itemize}
\item {} 
\sphinxAtStartPar
Statics and definition of fluids, as medium that has no shear stress at rest.

\item {} 
\sphinxAtStartPar
Kinematics

\item {} 
\sphinxAtStartPar
Dynamics

\item {} 
\sphinxAtStartPar
Models:
\begin{itemize}
\item {} 
\sphinxAtStartPar
Incompressible flows
\begin{itemize}
\item {} 
\sphinxAtStartPar
Governing equations, theorems and regimes of motion
\begin{itemize}
\item {} 
\sphinxAtStartPar
Inviscid

\item {} 
\sphinxAtStartPar
Irrotational

\end{itemize}

\end{itemize}

\item {} 
\sphinxAtStartPar
Compressible flows
\begin{itemize}
\item {} 
\sphinxAtStartPar
Inviscid

\item {} 
\sphinxAtStartPar
…

\end{itemize}

\end{itemize}

\end{itemize}

\sphinxstepscope


\chapter{Statics}
\label{\detokenize{ch/fluids/statics:statics}}\label{\detokenize{ch/fluids/statics:fluid-mechanics-statics}}\label{\detokenize{ch/fluids/statics::doc}}
\sphinxAtStartPar
The behavior of continuous medium in static conditions can be used to define a fluid.
\label{ch/fluids/statics:definition-0}
\begin{sphinxadmonition}{note}{Definition 1 (Fluid)}



\sphinxAtStartPar
A fluid can be defined as a continuous medium with no shear stress in static conditions. Thus, the stress tensor of an \sphinxstyleemphasis{isotropic fluid} under static conditions reads
\begin{equation*}
\begin{split}\mathbb{T}^s = - p \mathbb{I} \ ,\end{split}
\end{equation*}
\sphinxAtStartPar
where \(p\) is \sphinxstyleemphasis{pressure}. (\sphinxstylestrong{todo} mechanical? Thermodynamical?)
\end{sphinxadmonition}

\sphinxstepscope


\chapter{Constitutive Equations of Fluid Mechanics}
\label{\detokenize{ch/fluids/constitutive-equations:constitutive-equations-of-fluid-mechanics}}\label{\detokenize{ch/fluids/constitutive-equations:fluid-mechanics-constutive-equations}}\label{\detokenize{ch/fluids/constitutive-equations::doc}}

\section{Newtonian Fluids}
\label{\detokenize{ch/fluids/constitutive-equations:newtonian-fluids}}\label{\detokenize{ch/fluids/constitutive-equations:fluid-mechanics-constutive-equations-newtonian}}
\sphinxAtStartPar
A Newtonian fluid is the model of a fluid as a continuous medium whose stress tensor can be written as the sum of the hydrostatic pressure stress tensor \(-p \mathbb{I}\) \sphinxhyphen{} the only contribution holding in {\hyperref[\detokenize{ch/fluids/statics:fluid-mechanics-statics}]{\sphinxcrossref{\DUrole{std,std-ref}{statics}}}} \sphinxhyphen{} and a viscous stress tensor \(\mathbb{S}\)
\begin{equation*}
\begin{split}\mathbb{T} = -p \mathbb{I} + \mathbb{S} \ ,\end{split}
\end{equation*}
\sphinxAtStartPar
and the viscous stress tensor is isotropic and \sphinxstylestrong{linear} in the first\sphinxhyphen{}order spatial derivatives of the velocity field,
\begin{equation}\label{equation:ch/fluids/constitutive-equations:eq:stress-viscous}
\begin{split}\mathbb{S} = 2 \mu \mathbb{D} + \lambda (\nabla \cdot \vec{u}) \mathbb{I} \ ,\end{split}
\end{equation}
\sphinxAtStartPar
being \(\mu, \lambda\) the viscosity coefficients, and \(\mathbb{D}\) the strain velocity tensor \eqref{equation:ch/continuum/kinematics:eq:strain-vel-tensor}. Thus, the definition
\label{ch/fluids/constitutive-equations:definition-0}
\begin{sphinxadmonition}{note}{Definition 2 (Newtonian fluid)}



\sphinxAtStartPar
A Newtonian fluid is a continuous medium whose stress tensor reads
\begin{equation}\label{equation:ch/fluids/constitutive-equations:eq:stress-newtonian}
\begin{split}\mathbb{T} = - p \mathbb{I} + 2 \mu \mathbb{D} + \lambda (\nabla \cdot \vec{u}) \mathbb{I} \ .\end{split}
\end{equation}\end{sphinxadmonition}

\begin{sphinxadmonition}{note}{Note:}
\sphinxAtStartPar
The expression \eqref{equation:ch/fluids/constitutive-equations:eq:stress-viscous} of the viscosity stress tensor is the most general expression of a 2\sphinxhyphen{}nd order symmetric isotropic tensor proportional to 1\sphinxhyphen{}st order derivatives of a vector field.
\end{sphinxadmonition}

\sphinxstepscope


\chapter{Governing Equations of Fluid Mechanics}
\label{\detokenize{ch/fluids/governing-equations:governing-equations-of-fluid-mechanics}}\label{\detokenize{ch/fluids/governing-equations:fluid-mechanics-governing-equations}}\label{\detokenize{ch/fluids/governing-equations::doc}}

\section{Newtonian Fluid}
\label{\detokenize{ch/fluids/governing-equations:newtonian-fluid}}
\sphinxAtStartPar
The differential conservative form of the governing equations of a {\hyperref[\detokenize{ch/fluids/constitutive-equations:fluid-mechanics-constutive-equations-newtonian}]{\sphinxcrossref{\DUrole{std,std-ref}{Newtonian fluid}}}} directly follows from the {\hyperref[\detokenize{ch/continuum/principles-differential:continuum-principles-differential}]{\sphinxcrossref{\DUrole{std,std-ref}{governing equations of a continuum medium in differential form}}}},
\begin{equation*}
\begin{split}\begin{cases}
  \dfrac{\partial \rho }{\partial t} + \nabla \cdot \left( \rho \mathbf{u} \right) = 0 \\
  \dfrac{\partial }{\partial t} \left( \rho \mathbf{u} \right) +  \nabla \cdot \left( \rho \mathbf{u} \otimes \mathbf{u} \right) = \rho \mathbf{g} - \nabla \cdot \mathbb{T} \\
  \dfrac{\partial}{\partial t}  \left( \rho e^t \right) + \nabla \cdot \left( \rho e^t \mathbf{u} \right) = \rho \mathbf{g} \cdot \mathbf{u} + \nabla \cdot \left( \mathbb{T} \cdot \mathbf{u} \right) - \nabla \cdot \mathbf{q} + \rho r
\end{cases}\end{split}
\end{equation*}
\sphinxAtStartPar
using the expression \eqref{equation:ch/fluids/constitutive-equations:eq:stress-newtonian} of the stress tensor of a Newtonian fluid, and the required constitutive equations and equations of state characterizing the behavior of the medium and required to get a well\sphinxhyphen{}defined mathematical problem, providing the expression of thermodynamic variables and heat conduction flux as a function of the primary variables of the problem. As an example, Fourier’s law for heat conduction reads
\begin{equation*}
\begin{split}\vec{q} = - k \nabla T \ ,\end{split}
\end{equation*}
\sphinxAtStartPar
…

\sphinxstepscope


\chapter{Non\sphinxhyphen{}dimensional Equations of Fluid Mechanics}
\label{\detokenize{ch/fluids/dimensional-analysis:non-dimensional-equations-of-fluid-mechanics}}\label{\detokenize{ch/fluids/dimensional-analysis:fluid-mechanics-dimensional-analysis}}\label{\detokenize{ch/fluids/dimensional-analysis::doc}}\begin{equation*}
\begin{split}\begin{cases}
  \frac{}{}
\end{cases}\end{split}
\end{equation*}
\sphinxstepscope


\chapter{Incompressible Fluid Mechanics}
\label{\detokenize{ch/fluids/incompressible:incompressible-fluid-mechanics}}\label{\detokenize{ch/fluids/incompressible:fluid-mechanics-incompressible}}\label{\detokenize{ch/fluids/incompressible::doc}}
\sphinxAtStartPar
Chapter of a introductory course in incompressible fluid mechanics:
\begin{itemize}
\item {} 
\sphinxAtStartPar
statics

\item {} 
\sphinxAtStartPar
kinematics

\item {} 
\sphinxAtStartPar
governing equations

\item {} 
\sphinxAtStartPar
non\sphinxhyphen{}dimensional equations

\item {} 
\sphinxAtStartPar
vorticity dynamics

\item {} 
\sphinxAtStartPar
low\sphinxhyphen{}\(Re\) exact solutions

\item {} 
\sphinxAtStartPar
high\sphinxhyphen{}\(Re\) flows, incompressible inviscid irrotational flows:
\begin{itemize}
\item {} 
\sphinxAtStartPar
vorticity dynamics and Bernoulli theorems

\item {} 
\sphinxAtStartPar
aeronautical applications

\end{itemize}

\item {} 
\sphinxAtStartPar
boundary layer

\item {} 
\sphinxAtStartPar
instability and turbulence

\end{itemize}


\section{Navier\sphinxhyphen{}Stokes Equations}
\label{\detokenize{ch/fluids/incompressible:navier-stokes-equations}}\label{\detokenize{ch/fluids/incompressible:fluid-mechanics-incompressible-ns-eqn}}
\sphinxAtStartPar
The kinematic constraints (link to {\hyperref[\detokenize{ch/fluids/dimensional-analysis:fluid-mechanics-dimensional-analysis}]{\sphinxcrossref{\DUrole{std,std-ref}{Non\sphinxhyphen{}dimensional Equations of Fluid Mechanics}}}}?)
\begin{equation*}
\begin{split}\nabla \cdot \vec{v} = 0\end{split}
\end{equation*}
\sphinxAtStartPar
replaces mass balance in the governing equation and implies \(\frac{D \rho}{D t} = 0\), i.e. all the material particles have constant density in time.

\sphinxAtStartPar
If …
\begin{equation}\label{equation:ch/fluids/incompressible:eq:ns-eqn}
\begin{split}\begin{cases}
\rho \left[ \frac{\partial \vec{u}}{\partial t} + \vec{u} \cdot \nabla ) \vec{u} \right] - \mu \nabla^2 \vec{u} + \nabla P = \rho \vec{g} \\
\nabla \cdot \vec{u} = 0
\end{cases}\end{split}
\end{equation}
\sphinxAtStartPar
with the proper initial conditions, boundary conditions and \sphinxhyphen{} if required \sphinxhyphen{} {\hyperref[\detokenize{ch/fluids/incompressible:fluid-mechanics-incompressible-compatibility}]{\sphinxcrossref{\DUrole{std,std-ref}{compatibility conditions}}}}.

\phantomsection\label{\detokenize{ch/fluids/incompressible:fluid-mechanics-incompressible-compatibility}}\subsubsection*{Compatibility condition}

\sphinxAtStartPar
A compatibility condition is needed if the velocity field is prescribed on the whole boundary \(\partial V\) of the domain \(V\),
\begin{equation*}
\begin{split}\vec{u}\bigg|_{\partial V} = \vec{b}_n \ .\end{split}
\end{equation*}
\sphinxAtStartPar
The compatibility condition reads
\begin{equation*}
\begin{split}\oint_{\partial V} \vec{b}  \cdot \hat{n} = 0 \ ,\end{split}
\end{equation*}
\sphinxAtStartPar
to ensure that the conudary conditions are consistent with the incompressibility constraint, as it is readily proved using divergence theorem on the velocity field in \(V\),
\begin{equation*}
\begin{split}0 \equiv \int_V \underbrace{\nabla \cdot \vec{u}}_{= 0} = \oint_{\partial V} \hat{v} \cdot \hat{n} = \oint_{\partial V} \vec{b} \cdot \hat{n} \ .\end{split}
\end{equation*}

\section{Vorticity}
\label{\detokenize{ch/fluids/incompressible:vorticity}}\label{\detokenize{ch/fluids/incompressible:fluid-mechanics-incompressible-vorticity}}
\sphinxAtStartPar
A dynamical equation for vorticity \(\vec{\omega} := \nabla \times \vec{u}\) reailty follows taking the curl of Navier\sphinxhyphen{}Stokes equations \eqref{equation:ch/fluids/incompressible:eq:ns-eqn}
\begin{equation}\label{equation:ch/fluids/incompressible:eq:vorticity}
\begin{split}\frac{D \vec{\omega}}{D t} = (\vec{\omega} \cdot \nabla) \vec{u} + \nu \Delta \vec{\omega} \ ,\end{split}
\end{equation}
\sphinxAtStartPar
i.e. vorticity can be stretched\sphinxhyphen{}tilted by the term \((\vec{\omega} \cdot \nabla) \vec{u}\), or diffused by the term \(\nu \Delta \vec{\omega}\).

\sphinxAtStartPar
…


\section{Bernoulli theorems}
\label{\detokenize{ch/fluids/incompressible:bernoulli-theorems}}\label{\detokenize{ch/fluids/incompressible:id1}}
\sphinxAtStartPar
For an incompressible fluid, the advective term \((\vec{u} \cdot \nabla) \cdot \vec{u}\) can be recasted as
\begin{equation*}
\begin{split}(\vec{u} \cdot \nabla) \cdot \vec{u} = \vec{\omega} \times \vec{u} + \nabla \frac{|\vec{u}|^2}{2} \ ,\end{split}
\end{equation*}
\sphinxAtStartPar
so that the momentum equation in Navier\sphinxhyphen{}Stokes equations \eqref{equation:ch/fluids/incompressible:eq:ns-eqn} for fluids with uniform density \(\rho\) reads
\begin{equation}\label{equation:ch/fluids/incompressible:eq:ns-mom-bernoulli}
\begin{split} \rho \left[ \frac{\partial \vec{u}}{\partial t} + \vec{\omega} \times \vec{u} + \nabla \frac{|\vec{u}|^2}{2} \right] - \mu \Delta \vec{u} + \nabla P = \rho \vec{g} \ .\end{split}
\end{equation}
\sphinxAtStartPar
Starting from the form \eqref{equation:ch/fluids/incompressible:eq:ns-mom-bernoulli}, different forms of Bernoulli theorems are readilty derived with the proper assumptions.
\label{ch/fluids/incompressible:theorem-0}
\begin{sphinxadmonition}{note}{Theorem 1 (Bernoulli theorem along path and vortex lines in steady flows)}



\sphinxAtStartPar
In a steady incompressible inviscid flow with conservative volume forces, \(\vec{g} = - \nabla \chi\), the Bernoulli polynomial is constant along path (everywhere tangent to the velocity field, \(\hat{t}(\vec{r}) \parallel \vec{u}(\vec{r})\)) and vortex lines (everywhere tangent to the vorticity field, \(\hat{t}(\vec{r}) \parallel \vec{\omega}(\vec{r})\)), i.e. the directional derivative of the Bernoulli polynomial in the direction of the velocity or the vorticity field is identically zero,
\begin{equation*}
\begin{split}\hat{t} \cdot \nabla \left( \frac{|\vec{u}|^2}{2} + \frac{P}{\rho} + \chi \right) = 0 \ .\end{split}
\end{equation*}\end{sphinxadmonition}

\sphinxAtStartPar
The proof readily follows taking the scalar product with a unit\sphinxhyphen{}norm vector \(\hat{t}\) parallel to the local velocity or vorticity, and noting that \(\hat{t} \cdot \vec{u} \times \vec{\omega}\) is zero if either \(\hat{t} \parallel \vec{v}\) or \(\hat{t} \parallel \vec{\omega}\).
\label{ch/fluids/incompressible:theorem-1}
\begin{sphinxadmonition}{note}{Theorem 2 (Bernoulli theorem in irrotational inviscid steady flows)}



\sphinxAtStartPar
In a steady incompressible inviscid irrotational flow with conservative volume forces, \(\vec{g} = - \nabla \chi\), the Bernoulli polynomial is uniform in the whole domain, since its gradient is identically zero
\begin{equation*}
\begin{split}\nabla \left( \frac{|\vec{u}|^2}{2} + \frac{P}{\rho} + \chi \right) = 0 
\qquad \rightarrow \qquad \frac{|\vec{u}|^2}{2} + \frac{P}{\rho} + \chi = 0 \ .\end{split}
\end{equation*}\end{sphinxadmonition}
\label{ch/fluids/incompressible:theorem-2}
\begin{sphinxadmonition}{note}{Theorem 3 (Bernoulli theorem in irrotational inviscid flows)}



\sphinxAtStartPar
In an incompressible inviscid irrotational flow with conservative volume forces, \(\vec{g} = - \nabla \chi\), the Bernoulli polynomial is uniform in the connected irrotational regions of the domain \sphinxhyphen{} but not constant in time in general \sphinxhyphen{} , since its gradient is identically zero
\begin{equation*}
\begin{split}\nabla \left( \frac{\partial \phi}{\partial t} + \frac{|\vec{u}|^2}{2} + \frac{P}{\rho} + \chi \right) = 0 
\qquad \rightarrow \qquad \frac{\partial \phi}{\partial t} + \frac{|\vec{u}|^2}{2} + \frac{P}{\rho} + \chi = C(t) \ .\end{split}
\end{equation*}
\sphinxAtStartPar
being \(\phi\) the velocity potential used to write the irrotational velocity field as the gradient of a scalar function \(\vec{u} = \nabla \phi\).
\end{sphinxadmonition}

\begin{sphinxadmonition}{note}{Note:}
\sphinxAtStartPar
The assumption of inviscid flow is not directly required if irrotationality holds. Anyway the inviscid flow assumption may be required to make irrotationality condition holds. Lookinig at the vorticity equation \eqref{equation:ch/fluids/incompressible:eq:vorticity} the assumption of negligible viscosity prevents diffusion of vorticity from rotational regions to irrotational regions.
\end{sphinxadmonition}

\begin{sphinxadmonition}{note}{Note:}
\sphinxAtStartPar
A barotropic fluid is defined as a fluid where the pressure is a function of density only, \(P(\rho)\). For this kind of flows it’s possible to find a function \(\Pi\) so that
\begin{equation*}
\begin{split}d \Pi = \frac{d P}{\rho} \ .\end{split}
\end{equation*}
\sphinxAtStartPar
The results of this section derived for a uniform density flow hold for a barotropic fluid as well, replacing \(\frac{P}{\rho}\) with \(\Pi\).
\end{sphinxadmonition}

\sphinxstepscope


\chapter{Compressible Fluid Mechanics}
\label{\detokenize{ch/fluids/compressible:compressible-fluid-mechanics}}\label{\detokenize{ch/fluids/compressible:fluid-mechanics-compressible}}\label{\detokenize{ch/fluids/compressible::doc}}





\renewcommand{\indexname}{Proof Index}
\begin{sphinxtheindex}
\let\bigletter\sphinxstyleindexlettergroup
\bigletter{definition\sphinxhyphen{}0}
\item\relax\sphinxstyleindexentry{definition\sphinxhyphen{}0}\sphinxstyleindexextra{ch/fluids/statics}\sphinxstyleindexpageref{ch/fluids/statics:\detokenize{definition-0}}
\indexspace
\bigletter{theorem\sphinxhyphen{}0}
\item\relax\sphinxstyleindexentry{theorem\sphinxhyphen{}0}\sphinxstyleindexextra{ch/fluids/incompressible}\sphinxstyleindexpageref{ch/fluids/incompressible:\detokenize{theorem-0}}
\indexspace
\bigletter{theorem\sphinxhyphen{}1}
\item\relax\sphinxstyleindexentry{theorem\sphinxhyphen{}1}\sphinxstyleindexextra{ch/fluids/incompressible}\sphinxstyleindexpageref{ch/fluids/incompressible:\detokenize{theorem-1}}
\indexspace
\bigletter{theorem\sphinxhyphen{}2}
\item\relax\sphinxstyleindexentry{theorem\sphinxhyphen{}2}\sphinxstyleindexextra{ch/fluids/incompressible}\sphinxstyleindexpageref{ch/fluids/incompressible:\detokenize{theorem-2}}
\end{sphinxtheindex}

\renewcommand{\indexname}{Index}
\printindex
\end{document}